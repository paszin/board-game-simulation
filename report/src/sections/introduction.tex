%!TEX root=../report.tex

%grammarly checked

\section{Introduction}


Simulations are a problem-based discipline that allows for repeated testing of a hypothesis \cite{sokolowski2010modelingintro}. This statement covers two important aspects of simulations in general. The first key element in this statement is \textit{“problem-based”}. This means a simulation always addresses a question. The second key element is \textit{“repeated testing”}. This means a simulation is intended to run multiple times.


The field of simulations covers a wide range of applications.
Simulations, in particular, discrete-event simulations can be used to analyze the complex behavior of systems.
Examples are stock inventory monitoring, optimizing supply chains \cite{krenczyk2014production}, health care systems \cite{jacobson2006discrete, jun1999application} or games.

In order to simulate a real-world system, a model must be created. All observations are derived from the model. Understanding the foundations of how to build such models is the base for data-driven insights. It enables people to build a precise abstraction of reality, having a methodology to master complexity, understand required techniques and tools, and prove its validity by solid mathematical foundations \cite{sokolowski2010modelingintro}.

This report sets its focus on the simulation of board games, especially card games.

Board games are discussed by researchers in multiple branches of science like social science, computer science, mathematics, or psychology. Board games are either object of study or models for developing analogies \cite{gobet2004moves}. An example of an analogy is by Janssen who used the analysis of shredded card games to understand the evolution of institutional arrangements \cite{janssen2010evolution}.
If the board game becomes the object of research a simulation can be used to train a computer on how to play a game. A popular example is the development of AlphaGo \cite{silver2017mastering}. Another approach is to use simulations during the game design phase which is more challenging than gameplay. This can be done by tuning the game parameters or by modifying the rules \cite{hom2007automatic}.
To achieve the right balance of fun, time, or easiness of the game, parameters and rules must be chosen carefully. For example the number of cards at the beginning of a game or the number of jokers in the game. Using simulations makes it simple to collect quantitative data about little details that can have a huge impact on the game.

This report continues with the foundations of discrete-event simulation, followed by an example simulation of the card game \uno. Then, the design of \uno\ will be analyzed. After that I discuss the model of \uno\ and finally, I draw a conclusion.


%Simulations can play a role to understand player behavior.

% The large diversity of rules of shedding games show that this kind of game allows for tinkering of the rules and makes interesting variations of the same base game. Including or excluding a rule doubles the overall number of combinations. This means a game that is based on five rules could exist in 2^5 different variations.



%Board games are an interesting field for simulations because they consist of a clear set of rules and discrete events.


