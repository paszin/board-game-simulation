%!TEX root=../report.tex

\section{Introduction}

The field of simulations covers a wide range of applications.
Simulations, in particular discrete-event simulations can be used to analyze the complex behavior of systems. 
Applications range from stock inventory monitoring, optimizing supply chains, clinical something. \note{add citations}

In order to simulate real word system, a model must be created. All observations are derived from the model. Understanding the foundations of how to build such models is the base for data-driven insights. It enables people to build precise abstraction of reality, having a methodology to master complexity, understand required techniques and tools, and proof its validity by solid mathematical foundations. \cite{sokolowski2010modelingintro}

This report sets it focus to the simulation of board games, especially card games. Board games are an interesting field for simulations because they consist of a clear set of rules and discrete events. To achieve the right balance of fun, time, or easiness of the game, parameters must be chosen carefully. For example the number of cards at the beginning of the number of jokers in the game. Using simulations makes it simple to collect quantitative data about little details that can have a huge impact on the game. Simulations can play a role to understand player behavior. 

% The large diversity of rules of shedding games show that this kind of games allow for tinkering of the rules and make interesting variations of the same base game. Including or excluding a rule doubles the overall number of combinations. This means a game that is based on five rules could exist in 2^5 different variations.




Simulating board games can be seen as abstraction for other things: 
the evolution of institutional arrangements. \cite{janssen2010evolution}


