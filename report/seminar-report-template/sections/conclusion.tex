%!TEX root=../report.tex

\section{Conclusion}

I can confirm discrete-event simulation can be used to simulate card games. The overall implementation has just a few hundered lines of code. And with the architecture I presented I could easily implement two different versions of uno and even another card game that is absolutly unrelated to uno. 

The Institute of Industrial Engineers published a list of advantages and disadvantages of simulation and one of these is that simulation enable us to make better decisions, because every aspect can be tested. I can confirm this. I was able to look for games that take unexpectedly long and then visualize the game state to understand that a player could have one card left quite often till the game ends.
Another advantage is to understand how a system operates. While implementing rules, I noticed how precisely rules must be written down. So as a game developer I would always try to implement a simulation just to confirm my set of rules.
But there are also disadvantages of course. Results need an interpretation. And at some point it is unclear if one explanation is “my implemented players are not smart enough” or the game needs a modification.
And finally, I can confirm that 7 cards per player is a good rule. But if you want to add your own special card, then I recommend to run a simulation at first.


